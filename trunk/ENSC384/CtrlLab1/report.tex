\documentclass{report}

\begin{document}

\section{Step 1}

Figure X
shows speed response for inputs of 5, 15 and -15 v.

%Insert graph

\section{Step 2}

Figure x shows speed response for various motor friction parameters.
An input signal of 5V is used Viscous damping is assumed.
It can be seen that increasing friction both response speed and response magnitude decrease with increasing friction values.

%Insert graph

\section{Step 3}

The impact of motor viscosity was examined the output with a gear head.
The same final value was achieved seen however the response speed was slower.

%Insert graph

\section{Step 4}

The motor viscosity needed to reduce the motor speed to have of an ideal motor with no viscous friction
was experimentally found to be 0.0074.

%Insert graph

\section{Step 5}

Later

\section{Step 6}

For a given viscosity coefficient, the motor will always achieve a kown and final steady state sped.
As such, if the speed response is known for certain inertas, the actuall system inertai may be approximated.


\end{document}
