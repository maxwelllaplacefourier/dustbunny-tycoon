\section{Design Alternatives}

Truss design one (figure \ref{fig:hz1}) is based loosely on the first truss design found in lab one. 
This design was chosen as the horizontal truss component in the final structure. 
The criteria by which this decision was made is as as follows:

\begin{description}

\item[Overall simplicity] - Jigs were used to simplify truss assembly.
Design one has the fewest members, minimized the possibility of errors during construction.
%2/3:
Designs two and three both have twice as many members and more nodes than the first choice making them more difficult to construct and increasing the possibility of mistakes during cutting, grinding and soldering.

\item[Sufficiency] - % Might want a better word than Sufficiency
From previous calculations made in lab one, it was concluded that design one would be sufficient to support the horizontal force produced by motor acceleration.
	% This is dangerous to say and incorrect since we dont know what the "horizontal force produced by motor acceleration" is. We know it can take 5 N.
More complicated designs were deemed unnecessary.
%	This rational as a whole is more or less covered by other points

\item[Efficiency] - Design one was seen to have optimal efficiency when compared to the alternatives.
%2/3:
Truss two and truss three are 31\% and 63\% less efficient than the first design, which made them significantly less desirable given the design goal of optimizing efficiency. 

\item[Overall Length] - The first design had an overall length that was relatively small, allowing greater freedom to build the truss without concern for running out of material. 
%2/3:
Designs two and three had an overall length that was approximately 0.7 m longer than the first design, which would have left less room for error during construction.

\end{description}