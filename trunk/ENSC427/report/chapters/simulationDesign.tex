\chapter{Simulation Design} 

Stages - complete in order. 
See how many can be completed

\begin{enumerate}
\item Simulate a basic, fully connected, ad-hoc network

\item Simulate static and known partitions with one ferry on a well known route. 
For example, two partitions with a fixed number of nodes and one ferry traveling between the partitions on a known route.
This simplifies the packet routing algorithms.
\item Simulate a partitioned ad-hoc with dynamic topology (change partition boundaries, nodes) and random ferry paths.
\end{enumerate}

Network topology: multiple clients and one server communication (duplex communication)

\section{Basic Ad-Hoc Network}

\begin{itemize}
\item Ensure application and configuration profiles are working
\item Obtain baseline statistics (delay, jitter and loss)
\end{itemize}

\section{Partitioned Network With Message Ferrying}

Also known as story-carry-forward

Design scenarios based on parameters:

\begin{enumerate}
\item Degree of partitioning (high and low) - how many partitions are there
\item Distance - how many partitions data must cross
\item Number of ferries (few and many)
\item Ferry paths - random or defined to assist data transfer
\item Known network topology (partition locations, number of nodes, etc)
\end{enumerate}