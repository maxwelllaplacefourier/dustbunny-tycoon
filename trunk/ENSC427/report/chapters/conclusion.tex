\chapter{Conclusion} 

First and foremost, OPNET has been shown to be a suitable tool for analyzing message ferrying.
The node models created to analyze the specialized \lq{}state monitor\rq{} network were tested and validated.
A more complicated examination was performed with a network model involving ten source nodes and varying numbers of ferry and gateway nodes.
Statistics measuring update success and delay were defined, implemented and collected during an OPNET simulation.

%TODO: Transition

\section{Results}

A number of general conclusions can be drawn from the results presented in section \ref{sec:resultsChapter}.
Adding gateways and ferries was seen to reduces delay, reduces the memory requirements of ferries to achieve a desired success rate, and decreases variability in delay (see section \ref{sec:results_success_s2}).
As such, any message ferrying network should have a maximum number of ferries and gateways.
The success rate was seen to marginally improve when enabling source storage.
This improvement is expected to increase with additional ferries.
As such, it may be concluded that networks with few ferries need not implement this feature, however it should be enabled for networks with many ferries.

%NEED SOMETHING MORE

%An general discussion of results.
%Include a discussion on the feasibility of a practical implementation and what adoption threshold would be needed.

\section{Future Work}

There are four main categories for future work and improvements to the OPNET model in order to study task oriented message ferrying.

\subsection{Algorithm Improvements}

Many aspects of the ferrying algorithm implemented in this network are simplistic.
For example, there is no reverse communication from the gateway to message ferries indicating updates have been delivered and update messages may be discarded.
The implementation of an update acknowledgment mechanism could significantly improve performance and memory utilization.
Additionally, a more intelligent algorithm used by ferries to discard packets could also improve overall network performance.

\subsection{Model Improvements}

Many assumptions and simplifications were made when considering update and data transfer between nodes.
For example, near instantaneous data transfer, little to no packet loss, and a strict communication range of 60 meters was assumed.
Incorporating an existing point to point protocol for reliable wireless data transfer, such as WiFi or ZigBee would provide more realistic results.x

\subsection{Statistic Improvements}

Due to the unique nature of the network, common ways to measure statistics are not valid.
As such, custom logic was required to produce all statistics.
Measurement of only two statistics was implemented, delay and update success rate.
Adding additional statistics, such as number of active packets in the network and arrival order, would provide additional insights into the networks behaviour.

\subsection{Applicability and Network Model}

The simulations that were presented involves roughly even source node placement and random ferry movement.
It is unlikely that a real %change real
network would have these characteristics.
Creating and simulating a real-world network model and application would provide more realistic results.
