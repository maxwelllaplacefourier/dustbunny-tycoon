%we need to cite stuff here 
~\cite{adhocmsgferry}
~\cite{hybrid}
~\cite{Routing}
~\cite{wearable}
~\cite{QoSrouting}
~\cite{efficientrouting}
~\cite{implement}
~\cite{book1}
%slides -> report
%2 is 1
%1 is 3
%3 is 2

\chapter{Introduction} 

\section{Background}
Message ferrying is an approach of physically carrying data between network nodes which cannot communicate directly.
This is also known as "store-carry-forward" routing.~\cite{Routing}
Message ferrying can be of two types.  
They are message oriented ferries and task oriented ferries.~\cite{hybrid}
Message oriented ferries are the where the position of the ferries are controlled by an algorithm.  Task oriented ferries deals with random ferry movements.
Many of the previous research has focused on message ferrying within partitioned, wireless ad-hoc networks, which are message oriented ferries.~\cite{Routing}~\cite{adhocmsgferry}	
Our project research is in the category of task oriented ferries.
Some useful examples of applications can be seen in section \ref{sec:motivations}.


\subsection{Mobile Ad Hoc Networks}
A mobile ad hoc network (MANET),  is a self-configuring mesh network of mobile devices connected by wireless links.~\cite{book1}
These mobile devices are free to move independently in any direction and it acts as a router, where it must forward traffic unrelated to its own use.

\subsection{Partitioned Networks}
Partitioned networks are networks with no single hop or multiple hop route between some or even all node pairs.~\cite{hybrid}
In a partitioned network, nodes may remain fully disconnected or they may \emph{cluster}, forming subnetworks in which all nodes are connected.
All current used routing algorithms used in MANETs %TODO: list algorithms?
fail in the presence of partitioning.~\cite{Routing}

\subsection{Delay Tolerant Networks}
\label{sec:delay_loss_tolerant}
%This section is pretty lame
A delay tolerant network is one in which routing strategies and applications must tolerate significant delays delivering packets.
This delay may range from a few minutes up to hours or even days.~\cite{Routing}

%something more?

\subsection{Message Ferrying \& Store-Carry-Forward Routing}
\label{sec:ferrying_overview}
Message ferrying is a technique where mobile nodes in a MANET buffer data and physically carry it between nodes which are unable to communicate.~\cite{adhocmsgferry}
Store-carry-forward routing is a strategy which makes use of, typically, known or assigned trajectories of these mobile nodes, known as message ferries.
Some messages are dropped if no route to the destination can be found.  


%TODO HIPRI: ways message ferrying is used

\section{Motivations and Potential Applications}
\label{sec:motivations}

With the increasing numbers of mobile devices today, such as smartphones, laptops, tablets, netbooks, and more, there are many devices which could be potentially used as message ferries.~\cite{wearable} 
This project proposes one way to make use of the significant amount of technology we transport with us on a daily basis.
A message ferrying network could potentially transport small amounts of data over large distances essentially for free.
Beyond the use of message ferrying in remote sensor networks, discussed throughout this report, other applications of this technique might include tracking road traffic conditions, in-house utility management, automation for home devices, industrial monitoring, robot to robot communication and more.~\cite{book1}

%Some of these ideas are good but dont really fit in this section:
%The use of these mesh networks must have low-cost, very low power consumption,low data rate, cheap installation, flexibility, and increase connectivity to make it worthwhile in most cases.

\section{Project Goals}

Message ferrying has typically been examined within the context of improving throughput, reducing delay and increasing reliability within an ad hoc network.
Due to the complexity of incorporating message ferrying into existing ad hoc and MANET routing algorithms, this project will focus on a network in which data is transported strictly using ferries.
No clustering of network nodes and routing within subnetworks will be considered.
Surprisingly, very little research has been found for a network with these characteristics.
The goals of this project may be listed as follows.

\begin{itemize}
\item Design and implement a message ferrying algorithm.
\item Simulate this algorithm in a highly partitioned network without node clustering or subnetworks.
\item Evaluate the network considering topology and the message ferrying algorithm.
\item Examine the impact of node density, ferry count, and gateway count on the network.
\end{itemize}


%An overview look of our architecture:
%
%\begin{figure}[h]
%    \centering
%    \includegraphics[width=.5\textwidth]{images/wmn_general}
%    \caption{Wireless Mesh Network General Architecture Overview~\cite{book1}}
%    \label{fig:wmn_general}
%\end{figure}
%
%if something referring to this figure, use this \ref{fig:wmn_general}
