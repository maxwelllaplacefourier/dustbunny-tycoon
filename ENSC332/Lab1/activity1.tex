\section{Activity 1}

The first activity involved writing an assembly program which transfered a specific value to eight predefined memory locations.
The template given in the lab instructions was basis for creating the program.
A simplistic program was written which loaded the value to be copied to RAM into accumulator A.
It then used eight separate assembly instructions to write to each memory address.
After executing the eight store instructions, the program entered an infinite loop, effectively ending execution.
A potentially more efficient and elegant solution would used a counter use a single store command repeated eight times.
Regardless, the program performed as expected and using the debugging environment, the value of the RAM locations was seen to have the desired and expected value.

\subsection{Code}

{\footnotesize
\begin{lstlisting}
; Assembly Language Program
	ABSENTRY Entry ; absolute assembly application entry point
; Include derivative-specific definitions
	INCLUDE 'mc9s12dg256.inc'
ROMStart EQU $4000 ; absolute address to place my code/constant data
; $4000 is where code ROM starts for 9s12dx256 up to $7fff
; variable/data section
	ORG RAMStart ; RAMStart is defined in mc9s12dj256.inc as $1000
; Insert here your data definition.
Counter DS.W 1 ; set aside 1 word for counter in RAM
FiboRes DS.W 1
; code section
	ORG ROMStart
Entry:
	LDAA #$99 ; Load the value to be stored
	STAA $1220 ; Store it to 8 ram locations
	STAA $1221
	STAA $1222
	STAA $1223
	STAA $1224
	STAA $1225
	STAA $1226
	STAA $1227
HERE JMP HERE
	END ; if system has monitor program END can be removed
;***************************************************
;* Where to go when reset key is pressed *
;***************************************************
	ORG $FFFE
	DC.W Entry ; Reset Vector
\end{lstlisting}