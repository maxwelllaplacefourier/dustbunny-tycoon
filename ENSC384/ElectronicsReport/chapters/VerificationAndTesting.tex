\section{Verification and Testing}

This section details lab activities to test and verify the circuit. 
After the provided circuit components were soldered to the PCB, two tests were performed.
The first found the experimental transfer function between amplifer input voltage and voltage output to the motors.
The second test estimated the power dissipated by the MOSFET transistors.

\subsection{Amplifier Transfer Function}
\label{sec:tf}

To obtain the response of the motor to an applied input, the input voltage to the amplifier was adjusted using a potentiometer, 
Multimeters were used to capture to between the potentiometer output and ground (the amplifier input voltage input) and the associated voltage drop across the motor (output voltage).
During testing, an unloaded motor was connected to the amplifier.
By taking roughly thirty measurements for input voltages in the range of 0 to 4V, it was possible to establish the experimental relationship seen in figure 5, for a given circuit input voltage of 8V.

%TODO: Why so different

\subsection{Heat Sink Design}

Given the high-power nature of the MOSFET drive transistors, it is important to ensure they do not exceed their maximum operating temperature.
To this end, the maximum thermal resistance factor, $\theta_{JA}$ for which the MOSFETs may operate safely was calculated.
The thermal resistance factor is given by equation \ref{eq:theta}.

\begin{equation}
	\label{eq:theta}
	\theta_{JA} = {\frac{T_J - T_A} {P_D}}
\end{equation}

In this equation $T_A$ represents the ambient temperature, for which $25^{\circ}C$ was used. 
The $T_J$ value, or maximum silicon junction temperature, is supplied by the manufacturer and was seen to be $175^{\circ}C$ for both types of transistors.
% TODO: reference datasheet
The $P_D$ value is the power dissipated and is calculated by subtracting the power measured at the load from the power measured at the supply.
To determine the load power, two multimeters were used to measure voltage drop across the motor and current through the motor. 
During the test, the motor was loaded by attempting to hold the shaft.
The supply power was similarly calculated by measuring input voltage and current to the entire circuit.
It was assumed that that the power dissipation for components other than the MOSFETs was negligible.
For a loaded motor, it was seen that, Input: 10v @ 0.65 A, Output: 8.25v @ 0.70 A, producing the $P_D$ value in equation \ref{eq:PD} . 

\begin{equation}
	\label{eq:PD}
	P_D =  P_S - P_L = \left({10 V}{\times}{.65 A}) - ({8.25 V}{\times}{.70 A}\right)
	    { = 0.725 W }
\end{equation}

Based on the experimentally observed power dissipation, the maximum thermal junction resistance may be calculated.
This is shown in equation \ref{eq:theta2}, assuming all power is dissipated by a single MOSFET.
It should be noted that due to the H-bridge design, power dissipation is actually split across two different transistors.
The Junction-to-Ambient thermal resistance for both MOSFET transistors is less than  $63^{\circ}C/W$.
As such, it is clear that based on experimentally observed power dissipation, there is no need for a heat sink.

\begin{equation}
	\label{eq:theta2}
	\theta_{JA_{MAX}} = \left({175 - 25}\right) {0.725 }
	{ = 206  ^{\circ}C/W}
\end{equation}
