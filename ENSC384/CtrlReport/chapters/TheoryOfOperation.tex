\section{Theory of Operation}

\subsection{Determination of Truss Inertia}

\subsubsection{Essential Equations}
The truss inertia was calculated from the time domain equations for system speed-control.
Equation \ref{eq:w} is the general equation for DC motor speed control closed-loop response, with zero disturbance torque.

\begin{equation}
\label{eq:w}
w(t) = AK{\it K_m}\,{\it K_t}\, \left( 1-{{\rm e}^{-{\frac {t \left( {\it K_m}\,
{\it K_b}+{\it R_a}\,B+{\it K_t}\,{\it K_m}\,K \right) }{{\it Ra}\,
 \left( {\it J_{truss}}+{\it J_{mag}}+{\it J_{mot}} \right) }}}} \right) 
 \left( {\it K_m}\,{\it K_b}+{\it R_a}\,B+{\it K_t}\,{\it K_m}\,K \right) ^
{-1}
\end{equation}

Equation \ref{eq:B} is the equation for final velocity, $w_{fv}$, that has been solved for the unknown viscous damping constant, $B$.

\begin{equation}
\label{eq:B}
B =  \left( {\frac {AK{\it Km}\,{\it Kt}}{{\it w_fv}}}-{\it Km}\,{\it Kb}-{
\it Kt}\,{\it Km}\,K \right) {{\it Ra}}^{-1}
\end{equation}

By substituting \ref{eq:B} into \ref{eq:w}, the simplified equation for $w(t)$ is produced.

\begin{equation}
\label{eq:vel}
w(t) = {\it w_fv}\, \left( 1-{{\rm e}^{-{\frac {tAK{\it Km}\,{\it Kt}}{{\it 
w_fv}\,{\it Ra}\, \left( {\it Jtruss}+{\it Jmag}+{\it Jmot} \right) }}}}
 \right) 
\end{equation}

In \ref{eq:vel}, the variables that remain after substituting for constants are $t$, $w(t)$, $w_{fv}$, and $A$. 

\subsubsection{Determining the Variables}
Using a sampling time of 0.001s the velocity was recorded, and to be thorough, testing was done at $A$ values of $5V$, $7.5V$, and $9V$. By taking an average of $w(t)$ values (over a roughly one second period) after final velocity was achieved, a value for $w_{fv}$ was obtained. Finally, $t$ was chosen at $2/3$ final velocity, which provided a  corresponding $w(t)$ value.

By solving \ref{eq:vel} in terms of $J_{truss}$ and substituting the previous constant/variable values, three truss inertia values were calculated for two different arrangements, a) truss alone and b) full truss-magnet. By averaging the results, an experimental determination of $J_{truss}$ is achieved for each arrangement.

\subsection{Controller Design}

The transient response is characterized by behavior such as maximum overshoot, rise time, delay time, and settling time and parameters such as damping ratio ($\zeta$) and natural undamped frequency ($\omega_n$). 
The main objective is to reach the steady state stage as soon as possible. In order to achieve this goal there certain steps must be taken.

The response of the system becomes more oscillatory with larger percent overshoot as $\zeta$ decreases. 
Also setting time is affect by $\zeta$ and $\omega_n$ and a practical way to reduce ts is to increase $\omega_n$ while keeping $\zeta$ constant. 
However, this increased the overshoot which depends only on $\zeta$. 
To simplify our calculations, the assumption was made to ignore new zero in the transfer function numerator and to use conservative specifications.  %What new zero -> need to say adding derivative gain adds zero
	% TODO: Reference pole zero plot -> notice zero is further away.
By increasing or decreasing settling time, $\omega_n$ will decrease or increase accordingly (see Equation \ref{eq:ts}).

In addition, the overshoot changes as $\zeta$ increases or decreases. 
The gains $K_p$ and $K_d$ were calculated using $\zeta$ and $\omega_n$. 
These gains made a significant impact on the response of the system. 
For instance, $K_p$ increases overshoot while it has a small change of the setting time. 
Furthermore, $K_d$ has minor affect on both the settling time and overshoot. 
Therefore, to obtain better overall response it is important to add $K_d$ to improve overshoot.
Finally, the overall performance is generally measured by the step response and steady- state error.

The reason is that $K_i$ is not considered because there is no factors to cause steady-state error in the response such as disturbance torque which assumed to be zero and also it causes an increase in the settling time. 
Therefore, it was desired that unless integral gain was needed, it be left out.


%TODO: Fix Me.
%Effect of parameters and gains + how they relate.
%Ignoring new zero in transfer function - new assumption - use conservative specifications.
%Talk about where PO and Ts equations come from.
%Ts for $\zeta$ < 0.69 - just a heuristic so results are not perfectly accurate.

%TODO: Add equations, transfer function, second order system

%\begin{equation}
%	\label{eq:po}
%	{\it PO}={{\rm e}^{-{\frac {\pi \,\zeta}{\sqrt {1-{\zeta}^{2}}}}}}
%\end{equation}

%\begin{equation}
%	\label{eq:ts}
%	{\it T_s} ={\frac {3.2}{\zeta\,\omega_n}}
%\end{equation}

%Add pole zero plot.

%\subsection{Simulation}

%TODO: Fix Me.
%Is this subsection needed?
%Talk about what the simulation is doing?

\subsection{Amplifier Compensation}

A simulink model to implement software PWM including motor back EMF compensation and simple error correction feedback is shown in figure \ref{fig:swpwm}.
The core component of this model is the PWM generator; a ramp generator which outputs between zero and one and repeats ever 20 discreet steps . 
For a system timestep of 0.001 seconds, this gives the software PWM output a nominal frequency of 50 Hz.
Two feed back mechanisms have been implemented to improve output accuracy. 
The first is motor back EMF compensation.
This compensator estimates voltage generated by the spinning motor based on its encoder measured speed and back EMF constant (from Table \ref{tbl:sysparams}.
The second mechanism is error feedback using a proportional controller.
The input to this compensator is the actual voltage measured at the motor terminals.	
