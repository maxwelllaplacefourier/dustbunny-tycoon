\subsubsection{3D Structures}
%Discuss the limitations of using a 2D analysis to predict the behaviour of a 3D structure. How will the 3D structure behaviour differ from the 2D? What other concerns do you have for predicting the behaviour of the truss structure? How can you analyze those forces and deflections? (remember, the truss structure will be rotating about one end to move an object as quickly and accurately as possible from point A to point B)

%These sound more like modeling limitations.
%This should be 2d vs 3d (perhaps 3d modeling)

Although the results from 2D behavior resemble closely the 3D behaivor, 2D results are only approximation.
Several assumptions are made to simplify the calculations such as no bending, all joints are pinned and forces act on the same plane. 
The behavior of 3D is not exactly as predicted 2D behavior. However, 2D will give a good indication (approximation) of the magnitudes of the forces and deflection of each member in 3D design.
A 2D analysis will also likely provide a reliable indication of the most efficient 3D structure